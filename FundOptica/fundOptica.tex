%% Preamble %%
%% A minimal LaTeX preamble
%% Some packates are needed to implement
%% Asciidoc features

\documentclass[11pt,twocolumn,oneside]{article}
\setlength{\columnsep}{1cm}
%\setlength{\columnseprule}{thickness}
\usepackage[brazilian]{babel}
\usepackage[a4paper,
            %bindingoffset=0.2in,%
            left=3cm,right=3cm,top=3cm,bottom=3cm,%
            %footskip=.25in
            ]
            {geometry}
\usepackage[figurename=Fig.,font=footnotesize,labelfont=bf]{caption}
\usepackage{fancyhdr}

\pagestyle{fancy}
%\fancyhf{}
%\rhead{Overleaf}
%\lhead{Guides and tutorials}
\lfoot{\tiny{Prof. Jefferson Oliveira}}
%\cfoot{}
%\rfoot{\thepage}
%\usepackage{geometry}                % See geometry.pdf to learn the layout options. There are lots.
%\geometry{a4paper}               % ... or a4paper or a5paper or ...
%\geometry{landscape}                % Activate for for rotated page geometry
%\usepackage[parfill]{parskip}       % Activate to begin paragraphs with an empty line rather than an indent

\usepackage{tcolorbox}
\usepackage{lipsum}

\usepackage{epstopdf}
\usepackage{color}
% \usepackage[usenames, dvipsnames]{color}
% \usepackage{alltt}


\usepackage{amssymb}
% \usepackage{amsmath}
\usepackage{amsthm}
\usepackage[version=3]{mhchem}


% Needed to properly typeset
% standard unicode characters:
%
\RequirePackage{fix-cm}
\usepackage{fontspec}
\usepackage[Latin,Greek]{ucharclasses}
%
% NOTE: you must also use xelatex
% as the typesetting engine


% \usepackage{fontspec}
% \usepackage{polyglossia}
% \setmainlanguage{en}

\usepackage{hyperref}
\hypersetup{
    colorlinks=true,
    linkcolor=blue,
    filecolor=magenta,
    urlcolor=cyan,
}

\usepackage{graphicx}
\usepackage{wrapfig}
\graphicspath{ {images/} }
\DeclareGraphicsExtensions{.png, .jpg, jpeg, .pdf}

%% \DeclareGraphicsRule{.tif}{png}{.png}{`convert #1 `dirname #1`/`basename #1 .tif`.png}
%% Asciidoc TeX Macros %%


% \pagecolor{black}
%%%%%%%%%%%%


% Needed for Asciidoc

\newcommand{\admonition}[2]{\textbf{#1}: {#2}}
\newcommand{\rolered}[1]{ \textcolor{red}{#1} }
\newcommand{\roleblue}[1]{ \textcolor{blue}{#1} }

\newtheorem{theorem}{Theorem}
\newtheorem{proposition}{Proposition}
\newtheorem{corollary}{Corollary}
\newtheorem{lemma}{Lemma}
\newtheorem{definition}{Definition}
\newtheorem{conjecture}{Conjecture}
\newtheorem{problem}{Problem}
\newtheorem{exercise}{Exercise}
\newtheorem{example}{Example}
\newtheorem{note}{Note}
\newtheorem{joke}{Joke}
\newtheorem{objection}{Objection}





%%%%%%%%%%%%%%%%%%%%%%%%%%%%%%%%%%%%%%%%%%%%%%%%%%%%%%%

%  Extended quote environment with author

\renewenvironment{quotation}
{   \leftskip 4em \begin{em} }
{\end{em}\par }

\def\signed#1{{\leavevmode\unskip\nobreak\hfil\penalty50\hskip2em
  \hbox{}\nobreak\hfil\raise-3pt\hbox{(#1)}%
  \parfillskip=0pt \finalhyphendemerits=0 \endgraf}}


\newsavebox\mybox

\newenvironment{aquote}[1]
  {\savebox\mybox{#1}\begin{quotation}}
  {\signed{\usebox\mybox}\end{quotation}}

\newenvironment{tquote}[1]
  {  {\bf #1} \begin{quotation} \\ }
  { \end{quotation} }

%% BOXES: http://tex.stackexchange.com/questions/83930/what-are-the-different-kinds-of-boxes-in-latex
%% ENVIRONMENTS: https://www.sharelatex.com/learn/Environments

\newenvironment{asciidocbox}
  {\leftskip6em\rightskip6em\par}
  {\par}

\newenvironment{titledasciidocbox}[1]
  {\leftskip6em\rightskip6em\par{\bf #1}\vskip-0.6em\par}
  {\par}



%%%%%%%%%%%%%%%%%%%%%%%%%%%%%%%%%%%%%%%%%%%%%%%%%%%%%%%%

%% http://texblog.org/tag/rightskip/


\newenvironment{preamble}
  {}
  {}

%% http://tex.stackexchange.com/questions/99809/box-or-sidebar-for-additional-text
%%
\newenvironment{sidebar}[1][r]
  {\wrapfigure{#1}{0.5\textwidth}\tcolorbox}
  {\endtcolorbox\endwrapfigure}


%%%%%%%%%%

\newenvironment{comment*}
  {\leftskip6em\rightskip6em\par}
  {\par}

  \newenvironment{remark*}
  {\leftskip6em\rightskip6em\par}
  {\par}


%% Dummy environment for testing:

\newenvironment{foo}
  {\bf Foo.\ }
  {}


\newenvironment{foo*}
  {\bf Foo.\ }
  {}


\newenvironment{click}
  {\bf Click.\ }
  {}

\newenvironment{click*}
  {\bf Click.\ }
  {}  
  
\newenvironment{resposta*}
  {\bf Resposta:\\ }
  {}

\newenvironment{resposta}
  {\bf Resposta:\\ }
  {}  
  
\newenvironment{remark}
  {\bf Remark.\ }
  {}

\newenvironment{capsule}
  {\leftskip10em\par}
  {\par}

%%%%%%%%%%%%%%%%%%%%%%%%%%%%%%%%%%%%%%%%%%%%%%%%%%%%%

%% Style

\parindent0pt
\parskip8pt
%% User Macros %%
%% Front Matter %%

\title{Fundamentos da Óptica Geométrica}
\author{Jefferson Rodrigues de Oliveira}
\date{02-11-2020}


%% Begin Document %%

\begin{document}
\maketitle
\tableofcontents
\hypertarget{x-objetivos}{\section{Objetivos}}
Caro aluno, logo abaixo apresentarei \textbf{os principais objetivos que você deve alcançar} ao estudar este conteúdo:


\begin{itemize}

\item \textbf{Compreender} processos em que o fornecimento de calor a um sistema, ou corpo, pode produzir aumento de seu volume, resultando na realização de trabalho.

\item \textbf{Compreender} o primeiro princípio da termodinâmica: a quantidade de calor fornecida a um sistema é igual ao trabalho que ele realiza mais a variação de sua energia interna.

\item \textbf{Compreender} que o Primeiro Princípio da Termodinâmica expressa quantitativamente a Lei de Conservação da Energia.

\item \textbf{Saber aplicar} o Primeiro Princípio da Termodinâmica para resolver problemas envolvendo calor, trabalho e energia interna de um sistema.

\item \textbf{Compreender} que o funcionamento de máquinas térmicas requer sempre troca de calor entre duas fontes, uma quente e outra fria.

\item \textbf{Compreender} que, numa máquina térmica, só uma parte do calor fornecido é transformado em trabalho.

\end{itemize}


\hypertarget{x-raios-de-luz-e-feixes-de-luz}{\section{Raios de luz e feixes de luz}}
A Óptica Geométrica estuda a \textbf{propagação da luz} nos diferentes meios e os fenômenos que dela decorrem: a \textbf{reflexão} e a \textbf{refração}. Este estudo é feito a partir da noção de \textbf{raio de luz} e de \textbf{princípios fundamentais}.


Ondas de rádio, micro-ondas, radiações infravermelha e ultravioleta, luz, raios X, etc. São constituintes das chamadas \textbf{ondas eletromagnéticas}. A luz difere das demais ondas pelo fato de, ao incidir em nossas vidas, produzir as \textbf{sensações visuais}. Ou seja, a \textbf{luz} é o agente físico que, atuando nos órgãos visuais, é capaz de produzir a \textbf{sensação de visão}.


Para que um observador possa enxergar um corpo, seus olhos devem receber a luz que este corpo emite.


Para representar a luz emitida pela chama de uma vela que atinge a vista de um observador, utilizaremos linhas orientadas que fornecem a direção e o sentido de propagação da luz. Tais linhas são denominadas \textbf{raios de luz}.


Na prática, é impossível isolar um raio de luz, que, na verdade, é apenas uma representação gráfica da luz em propagação. O que realmente existe são os chamados \textbf{feixes de luz}, que representamos graficamente como um conjunto de raios de luz. Os feixes de luz podem ser \textbf{paralelos}, \textbf{divergentes} ou \textbf{convergentes}.


\hypertarget{x-fontes-de-luz}{\section{Fontes de luz}}
Todos os corpos que emitem luz são chamados de \textbf{fontes de luz}.


Podemos classificar estas fontes de acordo com a \textbf{emissão da luz}.


\begin{itemize}

\item \textbf{Fonte de luz primária (corpo luminoso)}: corpos que emitem a luz que eles produzem, ou seja, \textbf{emitem luz própria}. Exemplo: Sol, lâmpada elétrica acesa, chamas das velas, etc.

\item \textbf{Fonte de luz secundária (corpo iluminado)}: corpos que emitem a luz que recebem de outros corpos, ou seja \textbf{não produzem luz própria}. Exemplo: a Lua, que envia à Terra a luz que recebe do Sol, das paredes iluminadas por uma lâmpada elétrica, etc.

\end{itemize}


Podemos também classificar estas fontes de acordo com suas \textbf{dimensões}:


\begin{itemize}

\item \textbf{Fonte de luz pontual (puntiforme)}: são fontes cujas dimensões são desprezíveis em relação à distância que a separam dos outros corpos. Exemplo: A maioria das estrelas, apesar delas serem enormes, as distância que as separam do nosso planeta são muito maiores.

\item \textbf{Fonte de luz extensa}: são fontes cujas dimensões \textbf{não} são desprezíveis em relação à distância que a separam dos outros corpos. Exemplo: O Sol, observado da Terra.

\end{itemize}


Por fim, também podemos classificar estas fontes de acordo com suas cores:


\begin{itemize}

\item \textbf{Fonte de luz monocromática (simples)}: fonte de luz que apresentam apenas uma cor. Exemplo: a luz amarela emitida por lâmpadas de vapor de sódio.

\item \textbf{Fonte de luz policromática (composta)}: fonte de luz que resulta da superposição de luzes de cores diferentes. Exemplo: a luz solar (branca).

\end{itemize}


\hypertarget{x-velocidade-da-luz}{\subsection{Velocidade da luz}}
A velocidade da luz no vácuo é de $299792458\;m/s$, ou seja, aproximadamente $3,0\times 10^{8}\;m/s$. No \textbf{vácuo}, a luz apresenta \textbf{máxima velocidade}, independente da cor, ou seja, todas as cores apresentam a mesma velocidade igual a $c$. Entretanto, em meios materiais, as luzes monocromáticas apresentam velocidades diferentes, todas inferiores a $c$.


\hypertarget{x-ano-luz}{\subsection{Ano-luz}}
\textbf{Ano-luz} é a unidade de \textbf{comprimento} que corresponde à distância percorrida pela luz, no vácuo, durante um ano.


Para se ter uma ideia da dimensão do ano-luz, vamos transformá-lo em metros e depois em quilômetros. Imagine que no instante $t=0$ um novo raio de luz partiu do Sol e vai para o "infinito". Vamos acompanhá-lo durante 1 ano e medir a distância percorrida.
\begin{align*}
    1\;ano  &= 365,25\;dias \\
            &= 8776\;horas \\
            &= 525960\;min \\
            &= 31557600\;s \\
            &\approx 3,16\times 10^{7}\;s
\end{align*}
Utilizando a fórmula da velocidade média:
\begin{align*}
    v &= \dfrac{\Delta S}{\Delta t} \\
    c &= \dfrac{d}{t} \\
    d &= c \cdot t \\
    d &= 3,0\times 10^{8} \cdot 3,16\times 10^{7} \\
    d &\approx 9,5 \times 10^{15}
\end{align*}
Sendo assim, a velocidade da luz no vácuo é de aproximadamente $9,5\times 10^{15}\;m$ ou $9,5\times 10^{12}\;km$.


Esta unidade é bastante utilizada na \textbf{Astronomia}, devido às distância das estrelas até o nosso planeta.


\hypertarget{x-classificação-dos-meios}{\section{Classificação dos meios}}
\begin{itemize}

\item \textbf{Meios transparentes} são aqueles que permitem que a luz os atravesse descrevendo \textbf{trajetórias regulares e bem definidas}, ou seja, quando a luz atravessa o meio  e \textbf{permite} a visualização nítida dos objetos. O único meio absolutamente transparente é o vácuo, todavia, também podem ser considerados transparentes o ar atmosférico, a água pura, entre outros.

\item \textbf{Meios translúcidos} são aqueles em que a luz descreve \textbf{trajetórias irregulares com intensa difusão (espelhamento aleatório)}, provocada pelas partículas deste meio, ou seja, quando a luz atravessa o meio e \textbf{não permite} uma visão nítida dos objetos. É o que ocorre, por exemplo, quando a luz atravessa a neblina, o papel-manteiga, entre outros.

\item \textbf{Meios opacos} são aqueles através dos quais \textbf{a luz não se propaga}. Depois de incidir em um meio opaco, a luz é parcialmente absorvida e parcialmente refletida pelo meio. São opacos os seguinte meios: madeira, alvenaria, metais, entre outros.

\end{itemize}


Um meio em que todos os seus elementos de volume possuem as mesmas propriedades é denominado \textbf{homogêneo}. O vácuo é um meio homogêneo por excelência. O ar, em pequenas quantidades, pode ser considerado homogêneo. Mas a atmosfera com um todo não é homogênea.


\hypertarget{x-fenômenos-da-óptica-geométrica}{\section{Fenômenos da Óptica Geométrica}}
A óptica geométrica estuda, basicamente, trajetórias de luz em sua propagação. São de especial interesse nesse estudo dois fenômenos físicos fundamentais: a \textbf{reflexão} e a \textbf{refração}.


\begin{itemize}

\item \textbf{Reflexão} é o fenômeno que consiste no fato de a luz voltar a se propagar no meio de origem, após incidir na superfície de separação deste com outro meio.

\end{itemize}


Quando o feixe de luz incidente e refletido são paralelos entre si, denominamos de \textbf{reflexão regular}, é o que acontece, por exemplo, em superfícies lisas e polidas.


\begin{itemize}

\item \textbf{Refração} é o fenômeno que consiste no fato de a luz passar de um meio para outro diferente.

\end{itemize}


\hypertarget{x-a-cor-de-um-corpo}{\section{A cor de um corpo}}
A luz solar (ou a luz emitida por uma lâmpada fluorescente) é denominada \textbf{luz branca}.


A luz branca solar é \textbf{policromática}, ou seja, é composta por diversas cores, das quais podemos destacar sete: vermelho, alaranjado, amarelo, verde, azul, anil e violeta.


Por que quando iluminada pela luz do Sol, as folhas de uma árvore nos parecem verdes?


Porque essas folhas refletem de forma difusa para o meio a cpr componente verde e absorvendo as demais cores componentes da luz branca.


Vale ressaltar os seguintes pontos:


\begin{itemize}

\item Se vermos um corpo \textbf{branco,} é porque ele está \textbf{refletindo todas as cores} do espectro solar.

\item Se "vermos" um corpo \textbf{preto}, é porque ele está \textbf{absorvendo todas as cores} do espectro solar.

\item Um corpo que nos parece vermelho quando iluminado pela luz branca solar se apresentará escuro quando iluminado por luz monocromática de cor diferente da vermelha (azul, por exemplo).

\end{itemize}


\hypertarget{x-princípios-da-ótica-geométrica}{\section{Princípios da Ótica Geométrica}}
Os princípios da Óptica Geométrica são:


\begin{itemize}

\item \textbf{Princípio da propagação retilínea}:

\end{itemize}


Nos meios homogêneos e transparentes, a luz se propaga em linha reta.


Este princípio constitui a base para a explicação de diversos fenômenos, como, por exemplo, a formação de sombras e penumbras.


\begin{itemize}

\item \textbf{Princípio da independência dos raios de luz}:

\end{itemize}


Cada raio de luz se propaga em um meio, independentemente de qualquer outro raio.


Isto significa que, mesmo havendo cruzamento entre raios de luz, cada um segue seu caminho se nada tivesse acontecido.


\begin{itemize}

\item \textbf{Princípio da reversibilidade da luz}:

\end{itemize}


A trajetória seguida pela luz não depende do seu sentido de percurso,


Ou seja, se a luz faz um determinado percurso, é capaz de fazer o mesmo percurso em sentido inverso.


\hypertarget{x-sombra,-penumbra-e-eclipse}{\section{Sombra, penumbra e eclipse}}

\hypertarget{x-referências}{\section{Referências}}
CALÇADA, Caio Sérgio; SAMPAIO, José Luiz. Física Clássica: Termologia, Óptica e Ondas. Atual Editora, São Paulo, 2012.


CAMPOS, Bruna Manuele. "Entropia – O que é? Características, Fórmula, Exemplos e Exercícios". Gestão Educacional. Disponível em: \href{https://www.gestaoeducacional.com.br/entropia-o-que-e-caracteristicas/}{https://www.gestaoeducacional.com.br/entropia-o-que-e-caracteristicas/}


FERRARO, Nicolau Gilbert. "Termodinâmica". Blog Os Fundamentos da Física. Disponível em: \href{https://osfundamentosdafisica.blogspot.com/}{https://osfundamentosdafisica.blogspot.com/}


RAMALHO JR, Francisco; FERRARO, Nicolau Gilberto; SOARES, Paulo Antônio de Toledo. Os Fundamentos da Física vol. 2. \textbf{Moderna, São Paulo}, 2007.


VILLAS BÔAS, Newton; DOCA, Ricardo Helou; BISCUOLA, Gualter José. Tópicos de física, 2: termologia, ondulatória e óptica. \textbf{São Paulo: Saraiva}, 2012.


\end{document}

